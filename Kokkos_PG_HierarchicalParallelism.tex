
\chapter{Hierarchical Parallelism}\label{C:Hierarchical}
Node architectures on modern high-performance computers are characterized by ever more \emph{hierarchical parallelism}. 
A level in the hierachy is determined by the hardware resources which are shared between compute units at that level. 
Higher levels in the hierarchy also have access to all resources in its branch at lower levels of the hierarchy.
This concept is orthogonal to the concept of heterogeneity. 
For example, a node in a typical CPU-based cluster consists of a number of multicore CPUs.  Each core supports one or more hyperthreads, and each hyperthread can execute vector instructions.
This means there are 4 levels in the hierarchy of parallelism: 
\begin{enumerate}
\item CPU sockets share access to the same memory and network resources,
\item cores within a socket typically have a shared last level cache (LLC), 
\item hyperthreads on the same core have access to a shared L1 (and L2) cache and they submit instructions to the same execution units, and
\item vector units execute a shared instruction on multiple data items.
\end{enumerate}
GPU-based systems also have a hierarchy of 4 levels:
\begin{enumerate}
\item Multiple GPUs in the same node share access to the same host memory and network resources, 
\item core clusters (e.g. the SMs on an NVIDIA GPU) have a shared cache and access to the same high bandwidth memory on a single GPU, 
\item threads running on the same core cluster have access to the same L1 cache and scratch memory and they are 
\item grouped in so called Warps or Wave Fronts within which threads are always synchronous and can collaborate more closely for example via direct register swapping. 
\end{enumerate}
Kokkos provides a number of abstract levels of parallelism which at compile time or run time are mapped to the appropriate hardware features. 
This mapping is not necessarily static or predefined; it may differ for each kernel.
Furthermore, some mapping decisions happen at run time. 
This enables adaptive kernels which map work to different hardware resources depending on the work set size.
While Kokkos provides defaults and suggestions, the optimal mapping can be algorithm dependent. 
Hierarchical parallelism is accessible through execution policies.

\section{Thread Teams Basics}

Kokkos' most basic hierarchical parallelism concept is a thread team.
A \emph{thread team} is a collection of threads which can synchronize,
and which share a ``scratch pad'' memory
(see Section\ref {S:Hierarchical:Scratch}).

Instead of mapping a 1-D range of indices to hardware resources,
Kokkos' thread teams map a 2-D index range.
The first index is the \emph{league}, the index of the team.
The second index is the thread index within a team.
In CUDA this is equivalent to launching a 1-D grid of 1-D blocks.
The league size is arbitrary -- that is, it is only limited by the integer size type -- while the team size must fit in the hardware constraints.
As in CUDA, only a limited number of teams are actually active at the same time,
and they must run to completion before new ones are executed. 
Consequently it is not valid to use inter thread-team synchronization mechanisms
such as waits for events initiated by other thread teams. 

\subsection{Creating a Policy Instance}

Kokkos exposes use of thread teams with the \lstinline!Kokkos::TeamPolicy! execution policy.
To use thread teams you need to create a \lstinline|Kokkos::TeamPolicy| instance.
It can be created inline for the parallel dispatch call.
The constructors requires two arguments: a league size and a team size. 
As with the  \lstinline|Kokkos::RangePolicy| a specific execution tag and a specific execution space can be given as optional template arguments.
\begin{lstlisting}
// Using default execution space and launching 
// a league with league_size teams with team_size threads each
Kokkos::TeamPolicy<> 
        policy( league_size, team_size ); 

// Using  a specific execution space to 
// run a n_worksets x team_size parallelism
Kokkos::TeamPolicy<ExecutionSpace> 
        policy( league_size, team_size );

// Using a specific execution space and an execution tag 
Kokkos::TeamPolicy<SomeTag, ExecutionSpace> 
        policy( league_size, team_size ); 
\end{lstlisting}
 
\subsection{Basic Kernels}
Within the kernel the team policy member type provides the necessary functionality to use teams. 
It allows access to thread identifiers such as league rank and size and team rank and size, as well as team synchronous actions such as team barriers, reductions and scans.
\begin{lstlisting}
using Kokkos::TeamPolicy;
using Kokkos::parallel_for;

typedef TeamPolicy<ExecutionSpace>::member_type member_type;
// Create an instance of the policy
TeamPolicy<ExecutionSpace> policy (league_size, team_size);
// Launch a kernel
parallel_for (policy, KOKKOS_LAMBDA (member_type team_member) {
    // Calculate a global thread id
    int k = team_member.league_rank () * team_member.team_size () +
            team_member.team_rank ();
    // Calculate the sum of the global thread ids of this team
    int team_sum = team_member.reduce (k);
    // Atomicly add the value to a global value
    a() += team_sum;
  });
\end{lstlisting}

This approach of writing hierarchical parallelism algorithms makes it very explicit that a kernel using a \lstinline!TeamPolicy! constitutes a parallel region with respect to the team.

\section{Team Scratch Pads}\label{S:Hierarchical:Scratch}

Each Kokkos team has a ``scratch pad.''
This is an instance of a memory space accessible only by threads in that team.
Scratch pads let an algorithm load a workset into a shared space
and then collaboratively work on it with all members of a team.
The lifetime of data in a scratch pad is the lifetime of the team.
In particular, scratch pads are recycled by all logical teams running on the same physical set of cores.
During the lifetime of the team all operations allowed on global memory are allowed on the scratch memory.
This includes taking addresses and performing atomic operations on elements located in scratch space. 
Team-level scratch pads correspond to the per-block shared memory in Cuda,
or to the ``local store'' memory on the Cell processor.

Kokkos exposes scratch pads through a special memory space associated with the execution space:
\lstinline|execution_space::scratch_memory_space|.
You may allocate a chunk of scratch memory through the \lstinline|TeamPolicy| member type.
You may request multiple allocations from scratch, up to a user-provided maximum. 
The maximum is provided either through a function in the functor which returns a potentially team-size dependent value, 
or it can be specified as an argument to the constructor of the TeamPolicy. 
It is not valid to provide both values at the same time. 
The argument to the TeamPolicy can be used to set the shared memory size when using functors. 
One restriction on shared memory allocations is that they can not be freed during the lifetime of the team. 
This avoids the complexity of a memory pool,
and reduces the time it takes to obtain an allocation
(which currently is a few operations to calculate the offset). 

\begin{lstlisting}
template<class ExecutionSpace>
struct functor {
  typedef ExecutionSpace execution_space;
  typedef execution_space::member_type member_type; 

  KOKKOS_INLINE_FUNCTION
  void operator() (member_type team_member) const {
    size_t double_size = 5*team_member.team_size()*sizeof(double);

    // Get a shared team allocation on the scratch pad
    double* team_shared_a = (double*)
      team_member.team_shmem().get_shmem(double_size);

    // Get another allocation on the scratch pad
    int* team_shared_b = (int*)
      team_member.team_shmem().get_shmem(160*sizeof(int));

    // ... use the scratch allocations ...
  }

  // Provide the shared memory capacity.
  // This function takes the team_size as an argument,
  // which allows team_size dependent allocations.
  size_t team_shmem_size (int team_size) const {
    return sizeof(double)*5*team_size +
           sizeof(int)*160;
  }
};
\end{lstlisting}

Instead of simply getting raw allocations in memory, users can also allocate Views directly in scratch memory. 
This is achieved by providing the shared memory handle as the first argument of the View constructor.
Views also have a static member function which return their shared memory size requirements. 
The function expects the run-time dimensions as arguments, corresponding to View's constructor. 

\begin{lstlisting}
tyepdef Kokkos::DefaultExecutionSpace::scratch_memory_space
  ScratchSpace;
// Define a view type in ScratchSpace
typedef Kokkos::View<int*[4],ScratchSpace> shared_int_2d;

// Get the size of the shared memory allocation
size_t shared_size = shared_int_2d::shmem_size(team_size);
Kokkos::parallel_for(Kokkos::TeamPolicy<>(league_size,team_size),
                     KOKKOS_LAMBDA ( member_type team_member) {
  // Get a view allocated in team shared memory.
  // The constructor takes the shared memory handle and the 
  // runtime dimensions
  shared_int_2d A(team_member.team_shmem(), team_member.team_size());
  ...
      
});
\end{lstlisting}


\section{Nested Parallelism}

Instead of writing code which explicitly uses league and team rank indices, one can use nested parallelism to implement hierarchical algorithms.
Kokkos lets the user have up to three nested layers of parallelism.
The team and thread levels are the first two levels.
The third level is \emph{vector} parallelism.

You may use any of the three parallel patterns -- for, reduce, or scan -- at each level.
You may nest them and use them in conjunction with code that is aware of the league and team rank.
The different layers are accessible via special execution policies:
\lstinline|TeamThreadLoop| and \lstinline|ThreadVectorLoop|. 

\subsection{Team and Vector Loops}

The first nested level of parallel loops splits an index range over the threads of a team. 
This motivates the policy name \lstinline|TeamThreadRange|, 
which indicates that the loop is executed once by the team with the index range split over threads.
The loop count is not limited to the number of threads in a team, and how the index range is mapped to threads is architecture dependent.
It is not legal to nest multiple parallel loops using the \lstinline!TeamThreadRange! policy.
However, it is valid to have multiple parallel loops using the \lstinline!TeamThreadRange! policy follow each other in sequence, in the same kernel.  
Note that it is not legal to make a write access to POD data outside of the closure of a nested parallel layer. 
This is a conscious choice to prevent difficult to debug issues related to thread private, team shared and globally shared variables. 
A simple way to enforce this is by using the ``capture by value'' clause with lambdas. 
With the lambda being considered as \lstinline|const| inside the \lstinline!TeamThreadRange! loop,
the compiler will catch illegal accesses at compile time as a \lstinline|const| violation.  

The most simple use case is to have another \lstinline|parallel_for| nested inside a kernel. 
\begin{lstlisting}
Kokkos::parallel_for(Kokkos::TeamPolicy<>(league_size,team_size), 
                     KOKKOS_LAMBDA ( member_type team_member) {
  Scalar tmp;
  Kokkos::parallel_for(
    Kokkos::TeamThreadRange(team_member,loop_count), 
    [=] (int& i) {
    ....
    // tmp += i; // This would be an illegal access
  });
});
\end{lstlisting}

The \lstinline|parallel_reduce| construct can be used to perform optimized team-level reductions:

\begin{lstlisting}
using Kokkos::parallel_reduce;
using Kokkos::TeamPolicy;
using Kokkos::TeamThreadRange;
parallel_for (TeamPolicy<> (league_size, team_size),
                 KOKKOS_LAMBDA (member_type team_member) {
    // The default reduction uses Scalar's += operator
    // to combine thread contributions.
    Scalar sum;
    parallel_reduce (TeamThreadRange (team_member, loop_count), 
      [=] (int& i, Scalar& lsum) {
        // ... 
        lsum += ...;
      }, sum);

    // You may provide a custom reduction as another
    // lambda together with an initialization value.
    Scalar product;
    Scalar init_value = 1;
    parallel_reduce (TeamThreadRange (team_member, loop_count), 
      [=] (int& i, Scalar& lsum) {
        // ...
        lsum *= ...;
      }, product, [=] (Scalar& lsum, Scalar& update) {
        lsum *= update; 
      }, init_value);
  });
\end{lstlisting}

The third pattern is \lstinline|parallel_scan| which can be used to perform prefix scans.



\subsection{Execution Restriction with \lstinline|single|}

As stated in the earlier subsections, a kernel has to be treated as a parallel region with respect to threads (and vector lanes) within a team.
This means that global memory accesses outside of the respective nested levels potentially have to be protected against repetitive execution. 
A common example is the case where a team performs some calculation but only one result per team has to be written back to global memory. 
Kokkos provides the \lstinline|Kokkkos::single(Policy,Lambda)| function for this case.
The two policies \lstinline|Kokkos::PerTeam| and \lstinline|Kokkos::PerThread| restrict execution to once per thread or once per team. 

\begin{lstlisting}
using Kokkos::parallel_for;
using Kokkos::parallel_reduce;
using Kokkos::TeamThreadRange;
using Kokkos::ThreadVectorRange;
using Kokkos::PerThread;
// ...
parallel_for (TeamThreadRange (thread, 100), [=] (int i) {
    double sum = 0;
    // Perform a vector reduction with a thread
    parallel_reduce (ThreadVectorRange (thread, 100), 
      [=] (int i, double& lsum) {
        // ...
        lsum += ...;
      }, sum);
    // Add the result value into a team shared array.
    // Make sure it is only added once per thread.
    Kokkos::single (PerThread (), [=] () {
        shared_array(i) += sum;
      });
  });
  
double sum;
parallel_reduce (TeamThreadRange (thread, 99), 
  [=] (int i, double& lsum) {
    // Add the result value into a team shared array
    // Make sure its only added once per thread
    Kokkos::single (PerThread (thread), [=] () {
        lsum += someFunction (shared_array(i), 
                              shared_array(i+1));
      });
  }, sum);
  
// Add the per team contribution to global memory
Kokkos::single (PerTeam (thread), [=] () {
    global_array(thread.league_rank()) = sum;
  });

// ...
\end{lstlisting}
 

