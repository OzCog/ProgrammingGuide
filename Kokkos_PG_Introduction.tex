

\section{Introduction}
\label{C:intro}

\noindent
The field of High Performance Computing (HPC) is on the verge of entering a new era. The need
for a fundamental change comes from many angles including the growing acceptance that rates of
pure computation (often called FLOP/s) are a poor single optimization goal for scientific
workloads, as well as practical challenges in the form of producing energy efficient and cost
efficient processors. Since the convergence on the Message-Passing Interface (MPI) standard
in the mid 1990s, application developers have enjoyed a seemingly static view of the 
underlying machine --
that of a distributed collection of homogeneous nodes executing in collaboration. However, 
after almost two decades of domaince, the sole use of MPI to derive parallelism is acting as
a potential limiter to greater future performance. While we expect MPI to continue to function
as the basic mechanism for communication between compute nodes for the immediate future, 
additional parallelism is
likely to be required on the computing node itself if high performance and efficiency goal 
are to be realized.

In reviewing potential options for the computing nodes of the future the reader might fall
upon three broad categories of computing device: (1) the multi-core processor with powerful
serial performance, optimized to reduce operation latency; (2) many-core processors with
low to medium powered compute cores that are designed to extend the multi-core concept toward
throughput based computation and, finally, (3), the general-purpose graphics processor unit
(GP-GPU, or often, GPU) which is a much more numerous collection of low powered cores designed
to tolerate long latencies but provide performance through a much higher degree of 
parallelism and computational throughput. Any combination of these options might also be 
combined in the future.

The diversity of the options available for processor selection raises as interested question
as to how these should be programmed. In part due to their heritage, but also due to their
optimized designs, each of these hardware types offers a different programming solution and
a different set of guidelines to write applications to for highest performance. Options available
today include a number of shared memory approaches such as OpenMP, Cilk+, Thread Building Blocks
(TBB) as well as Linux p-threads. To target both contemporary multi/many-core processors and
GPUs technologies such as OpenMP, OpenACC and OpenCL might be used. Finally, for highest
performance on GPUs a 
programming model such as CUDA may be selected. Such a variety of options poses a problem 
to the application
developer of today which is reminiscent of the challenges before MPI became the default
communication library -- which model should be selected to provide portability across hardware
solutions, as well as provide high performance across each class of processor and
protect algorithm investment into the future. None of the models listed above have been able to 
provide practical solutions to these questions.

The Kokkos programming model described in this programming guide seeks to address these
concerns by providing an abstraction of both computation and application data allocation and
layout. These abstraction layers are designed specifically to isolate software developers from
fluctuation and diversity in hardware details yet provide portability and high levels of performance
across many architectures.

This guide provides an introduction to the motivations of developing such an abstraction library,
a coverage of the programming model and its implementation as an embedded C++ library requiring
no additional modifications to the base C++ language. As such it should be seen as an introduction
for new users and as a reference for application developers who are already employing Kokkos
in their applications. Finally, supplementary tutorial examples are included as part of the
Kokkos software release to help users experiment and explore the programming model through a 
gradual series of steps.

