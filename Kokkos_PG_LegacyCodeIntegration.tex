\chapter{Interoperability and Legacy Codes}

One goal of Kokkos is to support incremental adoption in legacy 
applications. This facilitates a step by step conversion allowing
for continuous testing of functionality (and in certain bounds) of 
performance. One feature of this is full interoperability with the 
underlying backend programming models. This also allows for 
target specific optimizations written directly in the backend model
in order to achieve maximal performance.

After reading this chapter, you should understand the following:
\begin{itemize}
\item Restriction on interoperability with raw OpenMP and Cuda.
\item How to handle external data structures.
\item How to incrementally convert legacy data structures.
\item How to call non-Kokkos third party libraries safely.
\end{itemize}
In all code examples in this chapter, we assume that all classes in
the \lstinline!Kokkos! namespace have been imported into the working
namespace.

\section{OpenMP, Pthreads and CUDA interoperability}

Since the implementation of Kokkos is achieved with a C++ library
it provides full interoperability with the underlying backend programming
models. In particular it allows for mixing of OpenMP, CUDA and Kokkos 
code in the same compilation unit. This is true for both the parallel 
execution layers of Kokkos as for the data layer. 

It is important to recognize that this does not lift certain restrictions. 
For example it is not valid to allocate a view inside of an OpenMP 
parallel region, the same way as it is not valid to allocate a View 
inside a \lstinline|parallel_for| kernel. Indeed there are things which 
are slightly more cumbersome when mixing the models. Assigning 
one view to another inside a Cuda kernel or an OpenMP parallel 
region is only possible if the destination view is unmanaged. 
During dispatch of kernels with parallel for  all Views referenced 
in the functor or lambda are automatically switched into unmanaged 
mode. This would not happen when simply entering an OpenMP 
parallel region. 

\subsection{Cuda interoperabiltiy}

The most important thing to know for Cuda interoperability is that the
\lstinline|KOKKOS_INLINE_FUNCTION| macro evaluates to 
\lstinline|__host__ __device__ inline|. This means that calling a pure
\lstinline|__device__| function (for example Cuda intrinsics or device 
functions of libraries) must be protected with the \lstinline|__CUDA_ARCH__|
pragma.

\begin{lstlisting}
__device__ SomeFunction(double* x) {
  ...
}

struct Functor {
  typedef Cuda execution_space;
  View<double*,Cuda> a;
  KOKKOS_INLINE_FUNCTION
  void operator(const int& i) {
    #ifdef __CUDA_ARCH__
    int block = blockIdx.x;
    int thread = threadIdx.x;
    if (thread == 0)
      SomeFunction(&a(block));
    #endif
  }
}
\end{lstlisting}

The \lstinline|RangePolicy| start a 1D grid of 1D thread blocks so that the index 
\lstinline|i| is calculated as \lstinline|blockIdx.x * blockDim.x + threadIdx.x|. 
For the \lstinline|TeamPolicy| the number of teams is the grid dimension, while
the number of threads per team is mapped to the Y-dimension of the Cuda 
thread-block. The optional vector length is mapped to the X-dimension. 
For example \lstinline|TeamPolicy<Cuda>(100,12,16)| would start a 1D grid of size 
100 with block-dimensions (16,12,1) while \lstinline|TeamPolicy<Cuda>(100,96)| would 
result in a grid size of 100 with block-dimensions of (1,96,1). The restrictions on the 
vector length (power of two and smaller than 32 for the Cuda execution space) 
guarantee that vector loops are performed by threads which are part of a single 
warp. 

\subsection{OpenMP}

One restriction on OpenMP interoperability is that it is not valid to increase
the number of threads via \lstinline|omp_set_num_threads()| after initializing 
Kokkos. This restriction is caused due to Kokkos allocating internal per thread
book keeping data structures. It is however valid to ask 
for the thread ID inside a Kokkos parallel kernel compiled for the OpenMP 
execution space. It is also valid to use OpenMP constructs such as OpenMP
atomics inside a parallel kernel or functions called by it. However it is undefined
what happens when mixing OpenMP and Kokkos atomics, since those will not 
necessarily map to the the same underlying mechanism. 


\section{Legacy data structures}

There are two principal mechanism to facilitate interoperability with legacy data 
structures: (i) Kokkos allocates data and raw pointers are extracted to create 
legacy data structures and (ii) unmanaged views can be used to view externally 
allocated data. In both cases it is mandatory to fix the Layout of the Kokkos view
to the actual layout used in the legacy data structure. Note that the user is responsible
for insuring proper access capabilities. For example a pointer obtained from a view
in the \lstinline|CudaSpace| may only be accessed from Cuda kernels, and a View
constructed from memory acquired through a call to \lstinline|new| will typically only 
be accessible from Execution spaces which can access the \lstinline|HostSpace|.

\subsection{Kokkos manages memory}

\subsubsection{Raw allocations}

A simple way to add support for multiple memory spaces to a legacy app is to 
use \lstinline|kokkos_malloc|, \lstinline|kokkos_free| and \lstinline|kokkos_realloc|.
The functions are templated on the memory space and thus allow targeted 
placement of data structures: 

\begin{lstlisting}
// Allocate an array of 100 doubles in the default memory space
double* a = (double*) kokkos_malloc<>(100*sizeof(double)); 

// Allocate an array of 150 int* in the Cuda UVM memory space
// This allocation is accessible from the host
int** 2d_array = (int**) kokkos_malloc<CudaUVMSpace>(150*sizeof(int*));

// Fill the pointer array with pointers to data in the Cuda Space
// Since it is not the UVM space you can access 2d_array[i][j] only inside a Cuda Kernel
for(int i=0;i<150;i++)
  2d_array[i] = (int*) kokkos_malloc<CudaSpace>(200*sizeof(int));
\end{lstlisting}

A common usage scenario of this capability is to allocate all memory in the CudaUVMSpace 
when compiling for GPUs. This allows all allocations to be accessible from legacy code sections
as well as from parallel kernels written with Kokkos. 

\subsection{External memory management}

When memory is managed externally, for example because Kokkos is used
in a library which is given pointers to data allocations as input, it can be necessary 
or convenient to wrap the data into Kokkos views. If the library anyway receives the 
data to create a copy it is straight forward to allocate the internal data structure as a
view and copy the data in a parallel kernel element by element. Note that you might
need to first copy into a host view before copying to the actual destination memory 
space:

\begin{lstlisting}
template<class ExecutionSpace>
void MyKokkosFunction(double* a, const double** b, int n, int m) {
  // Define the host execution space and the view types
  typedef HostSpace::execution_space host_space;
  typedef View<double*,ExecutionSpace> t_1d_device_view;
  typedef View<double**,ExecutionSpace> t_2d_device_view;
  
  // Allocate the view in the memory space of ExecutionSpace
  t_1d_device_view d_a("a",n);
  // Create a host copy of that view
  typename t_1d_device_view::HostMirror h_a = create_mirror_view(a);
  // Copy the data from the external allocation into the host view
  parallel_for(RangePolicy<host_space>(0,n),
    KOKKOS_LAMBDA (const int& i) {
    h_a(i) = a[i];
  });
  // Copy the data from the host view to the device view
  deep_copy(d_a,h_a);
  
  // Allocate a 2D view in the memory space of ExecutionSpace
  t_2d_device_view d_b("b",n,m);
  // Create a host copy of that view
  typename t_2d_device_view::HostMirror h_b = create_mirror_view(b);
  
  // Get the member_type of the team policy
  typedef TeamPolicy<host_space>::member_type t_team;
  // Run a 2D copy kernel using a TeamPolicy
  parallel_for(TeamPolicy<host_space>(n,m),KOKKOS_LAMBDA (const t_team& t) {
    const int i = t.team_rank();
    parallel_for(TeamThreadRange(t,m), [&] (const int& j) {
      h_b(i,j) = b[i][j];
    });
  });
  // Copy the data from the host to the device
  deep_copy(d_b,h_b);
}
\end{lstlisting}

Alternatively one can create a view which directly references the external allocation.
If that data is a multi dimensional view it is important to specify the Layout explicitly.
Furthermore all data must be part of the same allocation.

 \begin{lstlisting}
void MyKokkosFunction(double* a, const double* b, int n, int m) {
  // Define the host execution space and the view types
  typedef View<double*,DefaultHostExecutionSpace, Unmanaged> t_1d_view;
  typedef View<double**[3],LayoutRight, DefaultHostExecutionSpace, Unmanaged> 
     t_3d_view;
  
  // Create a 1D view of the external allocation
  t_1d_view d_a(a,n);
 
  // Create a 3D view of the second external allocation
  // This assumes that the data had a row major layout (i.e. the third index is stride 1)
  t_3d_view d_b(b,n,m);
}
\end{lstlisting}

\subsection{Views as the fundamental data owning structure}

An other option is to let Kokkos handle the basic allocations using Views, 
and then constructing the legacy data structures around them. Again it is
important to fix the Layout of the Views to whatever the layout of the legacy 
data was. 

\begin{lstlisting}
// Allocate a 2D view with row major layout
View<double**,LayoutRight,HostSpace> a("A",n,m);

// Allocate an array of pioneers
double** a_old = new double*[n];

// Fill the array with pointers to the rows of a
for(int i=0; i<n; i++)
  a_old[i] = &a(i,0);
\end{lstlisting}

\subsection{std::vector}

One of the most common data objects in C++ codes is \lstinline|std::vector|. 
Its semantics are unfortunately not compatible with Kokkos requirements 
and it is thus not well supported. A major problem is that functors and lambdas
are passed as const objects to the parallel execution. This design choice was made
to (i) prevent a common cause of race conditions and (ii) allow the underlying 
implementation more flexibility in how to share the functor and where to put it.
In particular this leaves the choice open for the implementation to give each thread 
an individual copy of the functor or place it in read only cache structures.

The semantics of \lstinline|std::vector| would in this case prevent a kernel from modifying its 
entries since a const \lstinline|std::vector| is read only. Furthermore creating multiple copies
of the functor would in deed replicate the vector data, since it has copy semantics.

Other issues with \lstinline|std::vector| are its unrestricted support for resize as well as push 
functionality. In a threaded environment support for those capabilities would bring
massive performance penalties. In particular access to the \lstinline|std::vector| would require
locks in order to prevent one thread to deallocate the view while another accesses its content.
And last but not least \lstinline|std::vector| is not supported on GPUs and thus would prevent
portability. 
 
Kokkos provides a drop in replacement for |std::vector| in the form of \lstinline|Kokkos::vector|.
Outside of parallel kernels its semantics are mostly the same as that of \lstinline|std::vector|, for
example assignments performs deep copies and resize and push functionality are provided.
One important difference is that it is valid to assign values to the elements of a const vector.

\section{Calling non-Kokkos libraries}

\section{Keeping track of data transfers}

 
