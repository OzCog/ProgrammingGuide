\chapter{View: Multidimensional array}\label{C:View}

After reading this chapter, you should understand the following:
\begin{itemize}
\item A Kokkos View is an array of zero or more dimensions
\item How to use View's first template parameter to specify the type
  of entries, the number of dimensions, and whether the dimensions are
  determined at run time or compile time
\item Kokkos handles array deallocation automatically
\item Kokkos chooses array layout at compile time for best overall
  performance, as a function of the computer architecture
\item How to set optional template parameters of View for low-level
  control of execution space, layout, and how Kokkos accesses array
  entries
\end{itemize}
In all code examples in this chapter, we assume that all classes in
the \lstinline!Kokkos! namespace have been imported into the working
namespace.

%%%%%%%%%%%%%%%%%%%%%%%%%%%%%%%%%%%%%%%%%%%%%%%%%%%%%%%%%%%%
\section{Why Kokkos needs multidimensional arrays}\label{S:View:Why}
%%%%%%%%%%%%%%%%%%%%%%%%%%%%%%%%%%%%%%%%%%%%%%%%%%%%%%%%%%%%

Many scientific and engineering codes spend a lot of time computing
with arrays of data.  Programmers invest a lot of effort making array
computations as fast as possible.  This effort is often intimately
bound to details of the computer architecture, run-time environment,
language, and programming model.  For example, optimal array layout
may differ based on the architecture, with a large integer factor
penalty if wrong.  Low-level issues like pointer alignment, array
layout, indexing overhead, and initialization all affect performance.
This is true even with sequential codes.  Thread parallelism adds even
more pitfalls, like first-touch allocation and false sharing.

For best performance, coders need to tie details of how they manage
arrays, to details of how parallel code accesses and manages those
arrays.  Programmers who write architecture-specific code then need to
mix low-level features of the architecture and programming model into
high-level algorithms.  This makes it hard to port codes between
architectures, or to write a code that still performs well as
architectures evolve.

Kokkos aims to relieve some of this burden, by optimizing array
management and access for the specific architecture.  Tying arrays to
shared-memory parallelism lets Kokkos optimize the former to the
latter.  For example, Kokkos can easily do first-touch allocation,
because it controls threads that it can use to initialize arrays.
Kokkos' architecture awareness lets it pick optimal layout and pad
allocations for good alignment.  Expert coders can also use Kokkos to
access low-level or more architecture-specific optimizations in a more
user-friendly way.  For instance, Kokkos makes it easy to experiment
with different array layouts.

%%%%%%%%%%%%%%%%%%%%%%%%%%%%%%%%%%%%%%%%%%%%%%%%%%%%%%%%%%%%
\section{Creating and using a View}\label{S:View:CreateUse}
%%%%%%%%%%%%%%%%%%%%%%%%%%%%%%%%%%%%%%%%%%%%%%%%%%%%%%%%%%%%

\subsection{Constructing a View}

A View is an array of zero or more dimensions.  Programmers set both
the type of entries, and the number of dimensions, at compile time, as
part of the type of the View.  For example, the following specifies
and allocates a View with four dimensions, whose entries have type
\texttt{int}:

\begin{lstlisting}
const size_t N0 = ...;
const size_t N1 = ...;
const size_t N2 = ...;
const size_t N3 = ...;
View<int****> a ("some label", N0, N1, N2, N3);
\end{lstlisting}

The string argument is a label, which Kokkos uses for debugging.
Different Views may have the same label.  The ellipses indicate some
integer dimensions specified at run time.  Users may also set some
dimensions at compile time.  For example, the following View has two
dimensions.  The first one (represented by the asterisk) is a run-time
dimension, and the second (represented by [3]) is a compile-time
dimension.  Thus, the View is an N by 3 array of type double, where N
is specified at run time in the View's constructor.

\begin{lstlisting}
const size_t N = ...;
View<double*[3]> b ("another label", N);
\end{lstlisting}

Views may have up to (at least) 8 dimensions, and any number of these
may be run-time or compile-time.  The only limitation is that the
run-time dimensions (if any) must appear first, followed by all the
compile-time dimensions (if any).  For example, the following are
valid three-dimensional View types:
\begin{itemize}
\item \lstinline!View<int***>!  (3 run-time dimensions)
\item \lstinline!View<int**[8]>!  (2 run-time, 1 compile-time)
\item \lstinline!View<int*[3][8]>!  (1 run-time, 2 compile-time)
\item \lstinline!View<int[4][3][8]>!  (3 compile-time)
\end{itemize}
and the following are \emph{not} valid three-dimensional View types:
\begin{itemize}
\item \lstinline!View<int[4]**>!
\item \lstinline!View<int[4][3]*>!
\item \lstinline!View<int[4]*[8]>!
\item \lstinline!View<int*[3]*>!
\end{itemize}
This limitation comes from the implementation of View using C++
templates.  View's first template parameter must be a valid C++ type.

\subsection{What types of data may a View contain?}

C++ lets users construct data types that may ``look like'' numbers in
terms of syntax, but do arbitrarily complicated things inside.  Some
of those things may not be thread safe, like unprotected updates to
global state.  Others may perform badly in parallel, like fine-grained
dynamic memory allocation.  Therefore it is strongly advised to use 
only simple data types inside Kokkos Views.

Users may always construct a View whose entries are 
\begin{itemize}
\item built-in data types (``plain old data''), like \texttt{int} or
  \texttt{double}, or
\item structs of built-in data types.
\end{itemize}

While it is in principal possible to have Kokkos Views of arbitrary 
objects there are a number of restrictions. The \lstinline|T| type 
must have a default constructor and destructor. For portability reasons
\lstinline|T| is not allowed to have virtual functions and one is not 
allowed to call functions with dynamic memory allocations inside of 
kernel code. Furthermore assignment operators as well as default constructor
and deconstructor must be marked with \lstinline|KOKKOS_[INLINE_]FUNCTION|. 
Keep in mind that the semantics of the resulting View are a combination of the 
Views 'view' semantics and the behaviour of the element type.

\subsection{Accessing entries (indexing)}

You may access an entry of a View using parentheses enclosing a
comma-delimited list of integer indices.  This looks just like a
Fortran multidimensional array access.  For example:
\begin{lstlisting}
const size_t N = ...;
View<double*[3][4]> a ("some label", N);
// KOKKOS_LAMBDA macro includes capture-by-value specifier [=].
parallel_for (N, KOKKOS_LAMBDA (const ptrdiff_t i) {
    const size_t j = ...;
    const size_t k = ...;
    const double a_ijk = a(i,j,k);
    /* rest of the loop body */
  });
\end{lstlisting}
Note how in the above example, we only access the View's entries in a
parallel loop body.  In general, you may only access a View's entries
in an execution space which is allowed to access that View's memory
space.  For example, if the default execution space is \lstinline!Cuda!, 
a View for which no specific Memory Space was fiven may not be accessed 
in host code \footnote{An exemption is if you specified for CUDA compilation
that the default memory space is CudaUVMSpace, which can be accessed from 
the host.}
Furthermore, access costs (e.g., latency and
bandwidth) may vary, depending on the View's ``native'' memory and
execution spaces, and the execution space from which you access it.
CUDA UVM may work, but it may also be slow, depending on your access
pattern and performance requirements.  Thus, best practice is to
access the View only in a Kokkos parallel for, reduce, or scan, using
the same execution space as the View.  This also ensures that access
to the View's entries respect first-touch allocation.  The first
(leftmost) dimension of the View is the \emph{parallel dimension},
over which it is most efficient to do parallel array access if the 
default memory layout is used (e.g. if no specific memory layout is 
specified).

\subsection{Memory management and copying}

\subsubsection{Reference counting}

Kokkos automatically manages deallocation of Views, through a
reference-counting mechanism.  Otherwise, Views behave like raw
pointers.  Copying or assigning a View does a shallow copy, and
changes the reference count.  (The View copied has its reference count
incremented, and the assigned-to View has its reference count
decremented.)  A View's destructor (called when the View falls out of
scope or during a stack unwind due to an exception) decrements the
reference count.  Once the reference count reaches zero, Kokkos may
deallocate the View.

For example, the following code allocates two Views, then assigns one
to the other.  That assignment may deallocate the first View, since it
reduces its reference count to zero.  It then increases the reference
count of the second View, since now both Views point to it.
\begin{lstlisting}
View<int*> a ("a", 10);
View<int*> b ("b", 10);
a = b; // assignment does shallow copy
\end{lstlisting}

For efficiency, View allocation and reference counting turn off inside
of Kokkos' parallel for, reduce, and scan operations.  This affects
what you can do with Views inside of Kokkos' parallel operations.

\subsubsection{Deep copy}

If you want to make a deep copy of a View's contents, you must do so
explicitly, using the \lstinline!deep_copy! function.  For example:
\begin{lstlisting}
View<int*> a ("a", 10);
View<int*> b ("b", 10);
deep_copy (a, b); // copy contents of b into a
\end{lstlisting}

\subsection{Where and how is the View allocated?}

\subsubsection{Execution and memory spaces}

Views are allocated by default, in the default execution space's
default memory space.  You may access the View's execution space via
its \lstinline!execution_space! typedef, and its memory space via its
\lstinline!memory_space! typedef.  You may also specify the memory space
explicitly as a template parameter.  For example, the following
allocates a View in CUDA device memory:
\begin{lstlisting}
View<int*, CudaSpace> a ("a", 100000);
\end{lstlisting}
and the following allocates a View in ``host'' memory, using the default 
host execution space for first-touch initialization:
\begin{lstlisting}
View<int*, HostSpace> a ("a", 100000);
\end{lstlisting}

Since there is no bijective association between execution spaces and memory 
spaces, Kokkos provides a way to explicitly provide both to a View as a 
\lstinline|Device|. 
\begin{lstlisting}
View<int*, Device<Cuda,CudaUVMSpace> > a ("a", 100000);
View<int*, Device<OpenMP,CudaUVMSpace> > b ("b", 100000);
\end{lstlisting}
In this cas \lstinline|a| and \lstinline|b| will live in the same memory space,
but \lstinline|a| will be initialized on the GPU while \lstinline|b| will be 
initialized on the host. 

\subsubsection{Initialization}

A View's entries are initialized to zero by default.  Initialization
happens in parallel, for first-touch allocation over the first
(leftmost) dimension of the View.

You may allocate a View without initializing.  For example:
\begin{lstlisting}
View<int*> x (ViewAllocateWithoutInitializing (label), 100000);
\end{lstlisting}
This is useful in situations where your dominant use of the View
exhibits a complicated access pattern. In this case it is best
to run the most costly kernel directly after initialization to 
execute the first touch pattern and get optimal memory affinity.

\subsection{How do I get the raw pointer?}

We discourage access to a View's ``raw'' pointer.  This circumvents
reference counting.  That is, the memory may be deallocated once the
View's reference count goes to zero, so holding on to a raw pointer
may result in invalid memory access.  Furthermore, it may not even be
possible to access the View's memory from a given execution space.
For example, a View in the \lstinline!Cuda! space points to CUDA device
memory. Also using raw pointers would normally defeat the usability 
of polymorpic layouts and automatic padding. 
Nevertheless, sometimes you really need access to the
pointer.  For such cases, we provide the \lstinline!ptr_on_device()!
method.  For example:
\begin{lstlisting}
// Legacy function that takes a raw pointer.
extern void legacyFunction (double* x_raw, const size_t len);

// Your function that takes a View.
void myFunction (const View<double*>& x) {
  // DON'T DO THIS UNLESS YOU MUST
  double* a_raw = a.ptr_on_device ();
  const size_t N = x.dimension_0 ();
  legacyFunction (a_raw, N);
}
\end{lstlisting}
A user is in most cases also allowed to obtain a pointer to a specific 
element via the usual \lstinline|&| operator. For example

\begin{lstlisting}
// Legacy function that takes a raw pointer.
void someLibraryFunction (double* x_raw);

KOKKOS_INLINE_FUNCTION
void foo(const View<double*>& x) {
  someLibraryFunction(&x[3]);
}
\end{lstlisting}

This is only valid if a Views reference type is an \lstinline|lvalue|.
That property can be queried statically at compile time from the view through
its \lstinline|reference_is_lvalue| member. 


%%%%%%%%%%%%%%%%%%%%%%%%%%%%%%%%%%%%%%%%%%%%%%%%%%%%%%%%%%%%
\section{Layout}\label{S:View:Layout}
%%%%%%%%%%%%%%%%%%%%%%%%%%%%%%%%%%%%%%%%%%%%%%%%%%%%%%%%%%%%

\subsection{Strides and dimensions}\label{SS:View:Layout:Strides}

\emph{Layout} refers to the mapping from a logical multidimensional
index $(i, j, k, \dots)$ to a physical memory offset.  Different
programming languages may have different layout conventions.  For
example, Fortran uses \emph{column-major} or ``left'' layout, where
consecutive entries in the same column of a 2-D array are contiguous
in memory.  Kokkos calls this \lstinline!LayoutLeft!.  C, C++, and Java use
\emph{row-major} or ``right'' layout, where consecutive entries in the
same row of a 2-D array are contiguous in memory.  Kokkos calls this
\lstinline!LayoutRight!.

The generalization of both left and right layouts is ``strided.''  For
a strided layout, each dimension has a \emph{stride}.  The stride for
that dimension determines how far apart in memory two array entries
are, whose indices in that dimension differ only by one, and whose
other indices are all the same.  For example, with a 3-D strided view
with strides $(s_1, s_2, s_3)$, entries $(i, j, k)$ and $(i, j+1, k)$
are $s_2$ entries (not bytes) apart in memory.  Kokkos calls this
\lstinline!LayoutStride!.

Strides may differ from dimensions.  For example, Kokkos reserves the
right to pad each dimension for cache or vector alignment.  You may
access the dimensions of a View using the \lstinline!dimension! method,
which takes the index of the dimension.

Strides are accessed using the \lstinline!stride! method.  It takes a raw
integer array, and only fills in as many entries as the rank of the View.
For example:
\begin{lstlisting}
const size_t N0 = ...;
const size_t N1 = ...;
const size_t N2 = ...;
View<int***> a ("a", N0, N1, N2);

int dim1 = a.dimension (1); // returns dimension 1
size_t strides[3]
a.strides (dims); // fill 'strides' with strides
\end{lstlisting}
You may also refer to specific dimensions without a runtime parameter:
\begin{lstlisting}
const size_t n0 = a.dimension_0 ();
const size_t n2 = a.dimension_2 ();
\end{lstlisting}

Users of the BLAS and LAPACK libraries may be familiar with the ideas
of layout and stride.  These libraries only accept matrices in
column-major format.  The stride between consecutive entries in the
same column is 1, and the stride between consecutive entries in the
same row is \lstinline!LDA! (``leading dimension of the matrix A'').  The
number of rows may be less than \lstinline!LDA!, but may not be greater.

\subsection{Other layouts}\label{SS:View::Layout::Other}

Other layouts are possible.  For example, Kokkos has a ``tiled''
layout, where a tile's entries are stored contiguously (in either row-
or column-major order) and tiles have compile-time dimensions.  One
may also use Kokkos to implement Morton ordering or variants thereof.

\subsection{Default layout depends on execution space}\label{SS:View:Layout:Default}

Kokkos selects a View's default layout for optimal parallel access
over the leftmost dimension, based on its execution space.  For
example, \lstinline!View<int**, Cuda>! has \lstinline!LayoutLeft!, so that
consecutive threads in the same warp access consecutive entries in
memory.  This \emph{coalesced access} gives the code better memory
bandwidth.  In contrast, \lstinline!View<int**, OpenMP>! has
\lstinline!LayoutRight!, so that a single thread accesses contiguous
entries of the array.  This avoids wasting cache lines and helps
prevent false sharing of a cache line between threads.

\subsection{Explicitly specifying layout}\label{SS:View:Layout:Explicit}

We prefer that users let Kokkos determine a View's layout, based on
its execution space.  However, sometimes you really need to specify
the layout.  For example, the BLAS and LAPACK libraries only accept
column-major arrays.  If you want to give a View to the BLAS or
LAPACK library, that View must be \lstinline!LayoutLeft!.  You may specify the
layout as a template parameter of View.  For example:
\begin{lstlisting}
const size_t N0 = ...;
const size_t N1 = ...;
View<double**, LayoutLeft> A ("A", N0, N1);

// Get 'LDA' for BLAS / LAPACK
int strides[2]; // any integer type works in stride()
A.stride (strides);
const int LDA = strides[1];
\end{lstlisting}
You may ask a View for its layout via its \lstinline!array_layout! typedef.
This can be helpful for C++ template metaprogramming.  For example:
\begin{lstlisting}
template<class ViewType>
void callBlas (const ViewType& A) {
  typedef typename ViewType::array_layout array_layout;
  if (std::is_same<array_layout, LayoutLeft>::value) {
    callSomeBlasFunction (A.ptr_on_device (), ...);
  } else {
    throw std::invalid_argument ("A is not LayoutLeft");
  }
}
\end{lstlisting}

%%%%%%%%%%%%%%%%%%%%%%%%%%%%%%%%%%%%%%%%%%%%%%%%%%%%%%%%%%%%
\section{Managing Data Placement}\label{S:View:Placement}
%%%%%%%%%%%%%%%%%%%%%%%%%%%%%%%%%%%%%%%%%%%%%%%%%%%%%%%%%%%%

TODO: Hey Christian, what did you mean by this?

Christian says: \lstinline!CudaSpace!, \lstinline!HostSpace!, etc.
Also, please emphasize memory space over execution space, and talk
about the UVM and pinned variants of \lstinline!CudaSpace!.

%%%%%%%%%%%%%%%%%%%%%%%%%%%%%%%%%%%%%%%%%%%%%%%%%%%%%%%%%%%%
\section{Access traits}\label{S:View:AccessTraits}
%%%%%%%%%%%%%%%%%%%%%%%%%%%%%%%%%%%%%%%%%%%%%%%%%%%%%%%%%%%%

\subsection{Const Views}

Different optional template parameters of View let you control how
Kokkos accesses entries of the array.  The simplest example involves
making a View \lstinline!const!, that is, making its entries read-only.  It
is a compile-time error to assign to an entry of a const View.  It
does not make sense to allocate a const View, but it does make sense
to assign a non-const View to a const View.  Here is an example:
\begin{lstlisting}
const size_t N0 = ...;
View<double*> a_nonconst ("a_nonconst", N0);

// Assign a nonconst View to a const View
View<const double*> a_const = a_nonconst;
// Pass the const View to some read-only function.
const double result = readOnlyFunction (a_const);
\end{lstlisting}
Note that a ``const View'' means the \emph{entries} are const.  You
may still assign to a const View.  \lstinline!View<const double*>!
corresponds exactly to \lstinline!const double*!, and
\lstinline!const View<double*>! to \lstinline!double* const!.

Const Views often allow the compiler to optimize more agressively by allowing it
to reason about possible write conflicts and data aliasing. For example in a vector
update \lstinline|a(i+1)+=b(i) with skewed indexing it is safe to vectorize if
\lstinline|b| is a View of const data.
  
\subsection{Memory Traits}

Another way to get optimized data accesses is to specify a memory trait. 
These traits are used to declare intended use of the particular view of 
an allocation. For example a particular kernel might use a view only for 
streaming writes. By declaring that intention Kokkos can insert the 
appropriate store intrinsics on each architecture if available. 

Currently 
In particular, Kokkos lets you declare that you plan
to access a read-only array irregularly (less spatial locality), using
the optional \lstinline!RandomAccess! template parameter.  You may only use
this with a const View.  For example:
\begin{lstlisting}
const size_t N0 = ...;
View<int*> a_nonconst ("a", N0); // allocate nonconst View
// Assign to const, RandomAccess View
View<const int*, RandomAccess> a_ra = a_nonconst;
\end{lstlisting}
For a View whose execution space is \lstinline!Cuda!, access to a
\lstinline!RandomAccess! View may use CUDA texture fetches.  Texture
fetches are not cache coherent with respect to writes, which is why
you must use read-only access.  The texture cache is optimized for
noncontiguous access, since it has a shorter cache line than the
regular cache.

\lstinline!RandomAccess! is valid for other execution spaces, and Kokkos
also requires a const View for those execution spaces.  Kokkos
reserves the right to perform analogous optimizations on other
execution spaces.

\subsection{Standard idiom for specifying access traits}

The standard idiom for View is to pass it around using as few template
parameters as possible.  Then, assign to a View with the desired
access traits only at the ``last moment,'' when those access traits
are needed just before entering a computational kernel.  This lets you
template C++ classes and functions on the View type, without
proliferating instantiations.  Here is an example:

\begin{lstlisting}
// Compute a sparse matrix-vector product, for a sparse
// matrix stored in compressed sparse row (CSR) format.
void
spmatvec (const View<double*>& y, 
  const View<const size_t*>& ptr,
  const View<const int*>& ind,
  const View<const double*>& val,
  const View<const double*>& x)
{
  // Access to x has less locality than access to y.
  View<const double*, RandomAccess> x_ra;
  typedef View<const size_t*>::size_type size_type;

  parallel_for (y.dimension_0 (), KOKKOS_LAMBDA (const size_type i) {
      double y_i = y(i);
      for (size_t k = ptr(i); k < ptr(i+1); ++k) {
        y_i += val(k) * x_ra(ind(k));
      }
      y(i) = y_i;
    });
}
\end{lstlisting}

\subsection{Unmanaged Views}

It's always better to let Kokkos control memory allocation, but
sometimes you don't have a choice.  You might have to work with an
application or an interface that returns a raw pointer, for example.
Kokkos lets you wrap raw pointers in an \emph{unmanaged View}.
``Unmanaged'' means that Kokkos does \emph{not} do reference counting
or automatic deallocation for those Views.  The following example
shows how to create an unmanaged View of host memory.  You may do this
for CUDA device memory too, or indeed for memory allocated in any
memory space, by specifying the View's execution or memory space
accordingly.
\begin{lstlisting}
// Sometimes other code gives you a raw pointer, ...
const size_t N0 = ...;
double* x_raw = malloc (N0 * sizeof (double));
{
  // ... but you want to access it with Kokkos.
  //
  // malloc() returns host memory, so we use the 
  // host memory space HostSpace.  Unmanaged
  // Views have no label, because labels work with
  // the reference counting system.
  View<double*, HostSpace, Unmanaged> x_view (x_raw, N0);

  functionThatTakesKokkosView (x_view);

  // It's safest for unmanaged Views to fall out of
  // scope, before freeing their memory.
}
free (x_raw);
\end{lstlisting}

\subsection{Atomic access}

The \lstinline!Atomic! access trait lets you create a View, such that every
read or write to any entry uses an atomic update.  This only works for
certain data types, for which the hardware and programming model
support atomic updates.  Currently, Kokkos limits atomic updates to
32-bit or 64-bit built-in types.
